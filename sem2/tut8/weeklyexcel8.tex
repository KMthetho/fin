\documentclass[a4paper,12pt]{article}
\usepackage{fancyhdr}
\usepackage{geometry}
\usepackage{multicol}
\usepackage{amsmath}
\usepackage{parskip}
\usepackage{amsfonts, amssymb, mathtools}

\geometry{top=1in, bottom=1in, left=3cm, right=3cm}

\pagestyle{fancy}
\fancyhf{}
\fancyhead[L]{Financial Mathematics Weekly Excel}
\fancyhead[R]{\thepage}

\fancyfoot[C]{\textit{“Remember, preparation is the key to success! Take your time, read each question carefully, and trust in your knowledge. Good luck!”}}

\title{\textbf{Financial Mathematics (Tutor Worksheet)}}
\author{}
\date{}

\begin{document}

\maketitle

\vspace{-0.5cm}
\noindent \textbf{Created By: K. Mthetho} \hfill \textbf{Total Marks: 10+10\(^{*}\)} \\
\textbf{Time: 1 hour} \hfill \textbf{Number of Pages: 2} \\
\textbf{May 2025} \hfill 

\vspace{0.5cm}

\section*{Instructions}

\begin{itemize}
    \item This sheet is compiled from past material with minor adjustments and mainly for your own practice. 
    \item Your workings are very important and earn part marks in excel assessments.
    \item Label columns and make sure your work is understandable.
    \item Aim to learn more than you already know.
    \item \textbf{Note:} The mark allocations in this sheet are not a true reflection of the actual marking standard.
\end{itemize}

\vfill

\noindent\rule{\textwidth}{0.4pt}
\begin{center}
\textit{Good luck and do your best! Remember, the goal is to test your own understanding and identify areas that need revision.}
\end{center}

\newpage

\begin{enumerate}
    \item[\textbf{Q1}] % adapted from TUTTEST8 19'
    A company is considering investing in a project. The project requires an initial investment of three payments, 
    each of R105 000. The first is due at the start of the project,
    the second six months later, 
    and the third payment is due one year after the start of the project.

    After 15 years, it is assumed that a major refurbishment of the infrastructure will be
    required, costing R200 000.
    
    The project is expected to provide a continuous income
    stream as follows:

    \begin{itemize}
        \item [*] R20 000 in the second year
        \item [*] R23 000 in the third year
        \item [*] R26 000 in the fourth year
        \item [*] R29 000 in the fifth year
    \end{itemize}

    Thereafter the continuous income stream is expected to increase by 3\% per annum
    (compound) at the start of each year. The income stream is expected to cease at the
    end of the 30th year from the start of the project.

    \begin{enumerate}
        \item Calculate the net present value of the project at a rate of interest of 8\% per annum
        effective.
        \hfill (6)
        \item Calculate the discounted payback period for the project, assuming a rate of interest of 8\% per annum effective. 
        \hfill (4)
    \end{enumerate}

    \hfill [\textbf{10}]

    \item[\textbf{Q2\(^{*}\)}] % adapted from CM1B Sep2021', maybe add in bonus Q(?)
    \begin{center}
        \underline{\textbf{BONUS QUESTION}}
    \end{center}
    \item[] 
    A company took out a loan of R750 000 repayable monthly in arrears over 30 years.
    The loan is repaid by a special compound increasing annuity in arrears. The first annual repayment in year 1 is \(X\). 
    Then total annual repayments increase at a rate of 7\% per annum up to the end of year 14, and at a rate of 10\% per annum thereafter.
    The repayment is calculated using a rate of interest of 8\% convertible quarterly.

    Calculate \(X\).

    \textit{Notice how the capital components can indeed be negative!}
    \hfill [\textbf{10}]
\end{enumerate}

\end{document}